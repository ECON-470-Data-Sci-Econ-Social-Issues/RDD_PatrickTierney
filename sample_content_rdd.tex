\textbf{Universal Child Benefit and Outcomes: \footnotesize \textsuperscript\ \href{Gonzlez-EffectUniversalChild-2013.pdf}{TThe Effect of a Universal Child Benefit on Conceptions, Abortions, and Early Maternal Labor Supply}, Libertad González, \textit{American Economic Journal: Economic Policy}, 2013}

\begin{itemize}
    \scriptsize

\item \textbf{Objective:} The article aims to explore the effects of a universal child benefit introduced in Spain in 2007 on fertility rates and maternal labor supply. It finds that the benefit led to an increase in fertility, mainly by reducing abortions, and influenced eligible mothers to extend their period out of the workforce after childbirth​​.

\item \textbf{Methodology \& Instrument:} Uses a regression discontinuity design to analyze the sharp cutoff established for benefit eligibility, comparing outcomes between treated and control families arbitrarily close to the cutoff to determine the treatment effect​​.

\item \textbf{Reason:} To leverage a natural experiment created by the unanticipated introduction of a universal child benefit in Spain in 2007. 

\item \textbf{Data:}  from the Spanish National Statistical Institute, which provides micro-data on all births in Spain, including information on weeks of gestation at birth. This data allows the study to estimate the date of conception with reasonable accuracy and analyze the impact of the introduction of the child benefit on the number of conceptions and abortions​​.

\item \textbf{Results:} The article's results indicate that the introduction of the child benefit in Spain was successful in increasing fertility, as evidenced by a significant rise in conceptions and a notable decrease in abortions, with the policy estimated to have increased the annual number of births by approximately 6%​​.

\end{itemize}