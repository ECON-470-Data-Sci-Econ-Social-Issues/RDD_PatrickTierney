\begin{frame}[t]

% Title at the top
\begin{block}{}
\centering
\maketitle
\end{block}

\begin{columns}[T] % align columns at the top

% Column 1
\begin{column}{.32\textwidth}
    \begin{block}{\Huge Abstract} % Very large section title


    \Large % Larger main text
The article aims to explore the effects of a universal child benefit introduced in Spain in 2007 on fertility rates and maternal labor supply. 
    \normalsize 
    \end{block}

    \vspace{1cm} % Add vertical space

    \begin{block}{\Huge Introduction} % Very large section title
    \Large % Larger main text
The study aims to understand whether such programs, which are prevalent in many countries and typically aim to encourage fertility and improve family well-being, are effective in achieving their intended goals.     \normalsize 
    \end{block}

    \vspace{1cm} % Add vertical space

    \begin{block}{\Huge Literature Review} % Very large section title
    \Large % Larger main text
   The Effect of the Universal Child Care Cash Benefit on Female Labour Supply in Spain:
   The article investigates the impact of Spain's universal child care cash benefit, known as the "baby bonus," on female labor supply. This policy, which offered a 2700 dollar subsidy per child born to increase the birth rate, is analyzed through a quasi-experiment using the Difference-in-Differences (DiD) method. Findings suggest the subsidy had a positive effect on increasing female labor participation by reducing child-related costs.
\begin{figure}
        \centering
        \includegraphics[width=0.5\linewidth]{Screenshot 2024-02-28 133743.png}
        \caption{Differences-in-Differences for Hours Worked Before and After policy}
        \label{fig:enter-label}
    \end{figure}
    \normalsize
    \end{block}
\end{column}

% Column 2
\begin{column}{.32\textwidth}
    \begin{block}{\Huge Methodology} % Highlighted block
    \Large % Larger main text
    Uses a regression discontinuity design to analyze the sharp cutoff established for benefit eligibility, comparing outcomes between treated and control families arbitrarily close to the cutoff to determine the treatment effect.
    \normalsize
    \end{block}

    \vspace{1cm} % Add vertical space

    \begin{block}{\Huge Findings} % Highlighted block
    \Large % Larger main text

    \Large
\begin{figure}
    \centering
    \includegraphics[width=0.5\linewidth]{Screenshot 2024-02-28 130327.png}
    \caption{Number of Abortions by Month 2005 -2009}
    \label{fig:enter-label}
    \begin{figure}
        \centering
        \includegraphics[width=0.5\linewidth]{output.png}
        \caption{Number of Conceptions by Month 2005 - 2009}
        \label{fig:enter-label}
    \end{figure}
\end{figure}
    \end{block}
\end{column}

% Column 3
\begin{column}{.32\textwidth}
    \begin{block}{\Huge Discussion} % Very large section title
    \Large % Larger main text
   The introduction of the child benefit led to a noticeable increase in the number of conceptions and a corresponding decrease in the incidence of abortions. This suggests that the policy was successful in encouraging new conceptions, resulting in a significant, estimated 6 percent increase in the annual number of births. 
    \normalsize 
    \end{block}

    \vspace{1cm} % Add vertical space

    \begin{block}{\Huge Conclusions} % Very large section title
    \Large % Larger main text
 These outcomes suggest that financial support policies can significantly influence family planning decisions and maternal employment patterns.
    \normalsize
    \end{block}

    \vspace{1cm} % Add vertical space

    \begin{block}{\Huge References} % Very large section title
    \Large % Larger main text
   González, Libertad. "The Effect of a Universal Child Benefit on Conceptions Abortions and Early Maternal Labor Supply." \textit{American Economic Journal: Economic Policy}, vol. 5, no. 3, Aug. 2013, pp. 160-188.
    \normalsize 
    \end{block}

    \vspace{1cm} % Add vertical space

    \begin{block}{\Huge Appendix} % Very large section title
    \Large % Larger main text
   \begin{figure}
       \centering
       \includegraphics[width=0.5\linewidth]{Screenshot 2024-02-28 132127.png}
       \caption{Day care expenditure by month of birth}
       \label{fig:enter-label}
       
   \end{figure}
    \normalsize 
    \end{block}
\end{column}

\end{columns}
    \vspace{1cm} % Add vertical space
% Custom Footer
\begin{beamercolorbox}[center]{section in head/foot}
    \Large 
\end{beamercolorbox}

\end{frame}
\end{document}\begin{frame}[t]\begin{frame}[t]